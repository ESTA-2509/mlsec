
\section{Conclusions}

%------------------------------------------------
\begin{frame}{Author's Conclusion}
\setstretch{1.5} % Set line spacing to 1.5
    \begin{itemize}
        \item \textbf{Fine-tuning improves robustness:}
        \begin{itemize}
            \item Fine-tuning models enhance performance and security under white-box FGSM attacks.
        \end{itemize}
        \item \textbf{Risks of Fine-tuning:}
        \begin{itemize}
            \item Fine-tuned models are more vulnerable to adversarial examples from their source models.
            \item Black-box attacks demonstrate increased risks when transfer learning is applied.
        \end{itemize}
        \item \textbf{Findings' Implication:}
        \begin{itemize}
            \item Highlights a trade-off between improved performance and susceptibility to attacks.
        \end{itemize}
    \end{itemize}
\end{frame}

%------------------------------------------------
\begin{frame}{Author's Conclusion}
\setstretch{1.5} % Set line spacing to 1.5
    \begin{itemize}
        \item \textbf{New Metrics:}
        \begin{itemize}
            \item Introduced a metric to evaluate the transferability of adversarial attacks.
        \end{itemize}
        \item \textbf{Future Implications:}
        \begin{itemize}
            \item Findings provide insights for designing transfer learning models that are robust and effective.
            \item Encourages further research into adversarial robustness in transfer learning.
        \end{itemize}
        \item \textbf{Call to Action:}
        \begin{itemize}
            \item Developers need to carefully consider potential risks in fine-tuned systems.
        \end{itemize}
    \end{itemize}
\end{frame}

%------------------------------------------------
\begin{frame}{Contributions of the Paper}
\setstretch{1.5} % Set line spacing to 1.5
    \begin{itemize}
        \item \textbf{Comprehensive Experiments:}
        \begin{itemize}
            \item Evaluated robustness of fine-tuned models under both white-box and black-box attacks.
        \end{itemize}
        \item \textbf{Novel Insights:}
        \begin{itemize}
            \item Demonstrated trade-offs between performance and security in transfer learning.
        \end{itemize}
        \item \textbf{New Evaluation Metrics:}
        \begin{itemize}
            \item Proposed a metric to assess the transferability of adversarial examples.
        \end{itemize}
        \item \textbf{Practical Implications:}
        \begin{itemize}
            \item Findings emphasize the importance of adversarial training and robust model design.
        \end{itemize}
    \end{itemize}
\end{frame}


% \begin{frame}{Equation}
%     Navier-Stokes Equations Expanded Form (3D):
%     \footnotesize
%         \begin{align*}
%             \rho\left(\frac{\partial u}{\partial t} + u\frac{\partial u}{\partial x} + v\frac{\partial u}{\partial y} + w\frac{\partial u}{\partial z}\right) &= -\frac{\partial p}{\partial x} + \mu\left(\frac{\partial^2 u}{\partial x^2} + \frac{\partial^2 u}{\partial y^2} + \frac{\partial^2 u}{\partial z^2}\right) + f_x \\[0.3cm]
%             \rho\left(\frac{\partial v}{\partial t} + u\frac{\partial v}{\partial x} + v\frac{\partial v}{\partial y} + w\frac{\partial v}{\partial z}\right) &= -\frac{\partial p}{\partial y} + \mu\left(\frac{\partial^2 v}{\partial x^2} + \frac{\partial^2 v}{\partial y^2} + \frac{\partial^2 v}{\partial z^2}\right) + f_y \\[0.3cm]
%             \rho\left(\frac{\partial w}{\partial t} + u\frac{\partial w}{\partial x} + v\frac{\partial w}{\partial y} + w\frac{\partial w}{\partial z}\right) &= -\frac{\partial p}{\partial z} + \mu\left(\frac{\partial^2 w}{\partial x^2} + \frac{\partial^2 w}{\partial y^2} + \frac{\partial^2 w}{\partial z^2}\right) + f_z
%         \end{align*}
        
%     where $\mathbf{v} = (u,v,w)$ is the velocity field, $p$ is the pressure, $\rho$ is the density, $\mu$ is the dynamic viscosity, and $\mathbf{f}$ represents external forces.
% \end{frame}

% %------------------------------------------------

% \begin{frame}[fragile]
%     \frametitle{Python}
    
%     \begin{python}
% def calcular_dobro(x):
%     """Retorna o dobro do número"""
%     return 2 * x

% # Testando a função
% numero = 5
% resultado = calcular_dobro(numero)
% print(f"O dobro de {numero} é {resultado}")
%     \end{python}
% \end{frame}

% %------------------------------------------------
% \begin{frame}[fragile]
%     \frametitle{Java}
    
%     \begin{java}
% public class Exemplo {
%     public static void main(String[] args) {
%         int numero = 5;
%         int dobro = 2 * numero;
        
%         System.out.println("O dobro de " + numero +
%                          " eh " + dobro);
%     }
% }
%     \end{java}
% \end{frame}
% %------------------------------------------------
